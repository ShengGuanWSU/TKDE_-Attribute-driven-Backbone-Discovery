%%
%% This is file `sample-acmsmall.tex',
%% generated with the docstrip utility.
%%
%% The original source files were:
%%
%% samples.dtx  (with options: `acmsmall')
%% 
%% IMPORTANT NOTICE:
%% 
%% For the copyright see the source file.
%% 
%% Any modified versions of this file must be renamed
%% with new filenames distinct from sample-acmsmall.tex.
%% 
%% For distribution of the original source see the terms
%% for copying and modification in the file samples.dtx.
%% 
%% This generated file may be distributed as long as the
%% original source files, as listed above, are part of the
%% same distribution. (The sources need not necessarily be
%% in the same archive or directory.)
%%
%% The first command in your LaTeX source must be the \documentclass command.
\documentclass[acmsmall]{acmart}

%%
%% \BibTeX command to typeset BibTeX logo in the docs
\AtBeginDocument{%
  \providecommand\BibTeX{{%
    \normalfont B\kern-0.5em{\scshape i\kern-0.25em b}\kern-0.8em\TeX}}}

%% Rights management information.  This information is sent to you
%% when you complete the rights form.  These commands have SAMPLE
%% values in them; it is your responsibility as an author to replace
%% the commands and values with those provided to you when you
%% complete the rights form.
\setcopyright{acmcopyright}
\copyrightyear{2018}
\acmYear{2018}
\acmDOI{10.1145/1122445.1122456}


%%
%% These commands are for a JOURNAL article.
\acmJournal{JACM}
\acmVolume{37}
\acmNumber{4}
\acmArticle{111}
\acmMonth{8}

%%
%% Submission ID.
%% Use this when submitting an article to a sponsored event. You'll
%% receive a unique submission ID from the organizers
%% of the event, and this ID should be used as the parameter to this command.
%%\acmSubmissionID{123-A56-BU3}

%%
%% The majority of ACM publications use numbered citations and
%% references.  The command \citestyle{authoryear} switches to the
%% "author year" style.
%%
%% If you are preparing content for an event
%% sponsored by ACM SIGGRAPH, you must use the "author year" style of
%% citations and references.
%% Uncommenting
%% the next command will enable that style.
%%\citestyle{acmauthoryear}

%%
%% end of the preamble, start of the body of the document source.
\begin{document}

%%
%% The "title" command has an optional parameter,
%% allowing the author to define a "short title" to be used in page headers.
\title{Extension for Attribute-driven backbone discovery}

%%
%% The "author" command and its associated commands are used to define
%% the authors and their affiliations.
%% Of note is the shared affiliation of the first two authors, and the
%% "authornote" and "authornotemark" commands
%% used to denote shared contribution to the research.
\author{Sheng Guan}
\authornote{}
\email{sxg967@case.edu}
\orcid{1234-5678-9012}
\authornotemark[1]
\email{webmaster@marysville-ohio.com}
\affiliation{%
  \institution{Case Western Reserve University}
  \streetaddress{P.O. Box 1212}
  \city{Dublin}
  \state{Ohio}
  \postcode{43017-6221}
}

\author{Yinghui Wu}
\affiliation{%
  \institution{Case Western Reserve University}
  \country{USA}}
\email{larst@affiliation.org}


%%
%% By default, the full list of authors will be used in the page
%% headers. Often, this list is too long, and will overlap
%% other information printed in the page headers. This command allows
%% the author to define a more concise list
%% of authors' names for this purpose.
\renewcommand{\shortauthors}{Sheng Guan and Yinghui Wu}

%%
%% The abstract is a short summary of the work to be presented in the
%% article.
\begin{abstract}
  Abstract
\end{abstract}

%%
%% The code below is generated by the tool at http://dl.acm.org/ccs.cfm.
%% Please copy and paste the code instead of the example below.
%%
\begin{CCSXML}
<ccs2012>
 <concept>
  <concept_id>10010520.10010553.10010562</concept_id>
  <concept_desc>Computer systems organization~Embedded systems</concept_desc>
  <concept_significance>500</concept_significance>
 </concept>
 <concept>
  <concept_id>10010520.10010575.10010755</concept_id>
  <concept_desc>Computer systems organization~Redundancy</concept_desc>
  <concept_significance>300</concept_significance>
 </concept>
 <concept>
  <concept_id>10010520.10010553.10010554</concept_id>
  <concept_desc>Computer systems organization~Robotics</concept_desc>
  <concept_significance>100</concept_significance>
 </concept>
 <concept>
  <concept_id>10003033.10003083.10003095</concept_id>
  <concept_desc>Networks~Network reliability</concept_desc>
  <concept_significance>100</concept_significance>
 </concept>
</ccs2012>
\end{CCSXML}

\ccsdesc[500]{Computer systems organization~Embedded systems}
\ccsdesc[300]{Computer systems organization~Redundancy}
\ccsdesc{Computer systems organization~Robotics}
\ccsdesc[100]{Networks~Network reliability}

%%
%% Keywords. The author(s) should pick words that accurately describe
%% the work being presented. Separate the keywords with commas.
\keywords{datasets, neural networks, gaze detection, text tagging}


%%
%% This command processes the author and affiliation and title
%% information and builds the first part of the formatted document.
\maketitle



%%%%%%%%%%%%%%%%%%% Section 8 %%%%%%%%%%%%%%%%%%%%%%%
\IEEEraisesectionheading{\section{Introduction}
\label{sec:introduction}}

\IEEEPARstart{T}{his} is introduction part. 
%%%%%%%%%%%%%%%%%%% Section 8 %%%%%%%%%%%%%%%%%%%%%%%
\vspace{-1ex}
\section{Attribute-driven Backbones}
\label{sec-pre}

\subsection{Graphs and Attributed Backbones}

\stitle{Graphs}. We consider an \eat{a directed}
attributed graph
$G$ = $(V,E,F_A)$ with a finite set of
nodes $V$, and
a finite set of edges $E\subseteq V\times V$.
Each node $v\in V$ has a
{\em node tuple} $F_A(v)$ =
$\{(A_1, a_1), \ldots, (A_n,a_n)\}$
defined on a set of node attributes $\A$,
where a pair $(A_i, a_i)\in F_A(v)$
states that the attribute $v.A_i\in\A$ has
a value $a_i\in\adom(A_i)$. Here
(a) $\A$ refers to a set of all the node
attributes seen in $G$; and (b)
$\adom(A_i)$ is a finite {\em active domain} of
attribute $A_i$ in $G$, and contains
all the values of $v.A_i$, where $v$ ranges
over all the nodes in $V$.


We do not assume that all the nodes in $G$
have the same set of attributes. Specifically,
 we denote the set of all the attributes in $F_A(v)$
as $\A(v)$ ($\A(v)\subseteq\A$).


\stitle{Attribute-driven Backbone}.
Given a graph $G$ = $(V,E,F_A)$, an
attribute-driven backbone $T$
is a tree $(V_T,E_T,F_T)$, where
\vspace{-.75ex}
\bi
\item $E_T\subseteq E$, and $V_T\subseteq V$ are the nodes that are the end nodes of
the edges in $E_T$, \ie $T$ is an edge-induced subtree of $G$;
\item Each edge $e$=$(v,v')\in E_T$
has a set of {\em affinitive} attributes
$F_T(e)\subseteq\A(v)\cap\A(v')$.
\ei

Intuitively, affinity attributes $F_T(e)$
for an edge $e$ indicates possible
common attributes of its two end nodes
that can determine the strength of
$e$. A common example is that friendship
is more likely to be formed between two
social network users who share common interests
(\eg 'hobby', 'activity', 'fan').

\vspace{.5ex}
In the following sections, we shall refer to
attribute-driven backbones as ``backbones''
for simplicity.

\begin{example}
Two backbones $T_1$ and
$T_2$ are
illustrated in Fig.~\ref{exa-motivate}.
For edge $(v_5,v_6)$ in $T_1$,
$F_{T_1}((v_5, v_6))$ = $\{\texttt{location}\}$,
stating that $v_5$ and $v_6$ are geographically close to each other
in $T_1$. Other possible affinitive
attributes for $((v_5,v_6))$
can be $\{\texttt{tour\_type}\}$ or
 $\{\texttt{tour\_type}, \texttt{location}\}$.
Similarly, the affinitive attributes
for edge $(v_1, v_8)$ in $T_2$ contains
\texttt{location} and \texttt{attr\_type},
suggesting that they are close
in terms of location and both host
tourists with common interests.
\end{example}


\vspace{-.5ex}
\stitle{Cost model}. Given a set of {\em interested nodes} $V_I\subseteq V$,
one wants to to
find backbones that can cover
$V_I$ with small edge cost. We identify
a cost model for backbones.

\etitle{Node interestingness}. Given a node $v\in V_I$,
we consider a function $I$ that quantifies the interestingness $I(v)\in(0,1]$
of $v$. In practice, $I$ can be
specified by relevant entity search in graphs~\cite{lissandrini2018x}. For example, 
a user may specify a term 
``art meseum'' and identify nodes of interests as 
the set {$V_I$ = $\{v_3, v_8,v_9,v_{10},v_{12}\}$
with relevance scores as their
interestingness (Fig.~\ref{fig:backboneex}).

\etitle{Edge cost}. We consider a cost function $C$: $E$ $\times \A$ $\rightarrow \mathbb{R}^{+}$
that computes a non-negative connection cost for
an edge $e = (v, v')$ in the context of
its affinitive attributes $F_T(e)$.
Given a backbone edge $e\in E_T$ and its
affinitive attributes
$F_T(e)$, we
define $\C(e, F_T(e))$ in the form of
\[
C(e, F_T(e)) = 1-
\sum
_{A\in F_T(e)}\frac{\ksim(F_A(v), F_A(v'))}{|F_T(e)|}
\]
where $\ksim(F_{A_i}(v), F_{A_i}(v'))$ \eat{$\ksim(F_{A_i}(v), F_{A_i}(v')))$}quantifies the
closeness between the values of $v.A$ and $v'.A$
given attributes $A$ and $A'$.
The closeness can be
specified by established similarity measures~\cite{augsten2013similarity},
semantic closeness,
or geographical distance.
%%%%%%%%%%%%%
The function $C$ can capture
tuple similarity by aggregating
attribute similarity independently.

\stitle{Cost of Attribute-driven Backbone}.
We now introduce a cost measure for backbones.
Given a graph $G$ = $(V,E,F_A)$,
a set of interested nodes $V_I\subseteq V$,
measures $I$ and edge cost model $C$, the
{\em cost} of a backbone $T$=$(V_T, E_T, F_T)$ in $G$, denoted by $\cost(T)$,
is defined as
\[
\cost(T) = \sum_{e\in E_T}C(e, F_T(e)) +  \sum_{v\in V_I\setminus V_T}I(v)
\]

The cost model penalizes
the total edge cost in $T$ (first term), as well as the
``lost'' interestingness from the nodes in $V_I$
that are not covered by $V_T$  (second term).
The smaller \eat{$C(T)$} $\cost(T)$ is, the better $T$ should be.
An optimal case for a backbone $T$ is a tree
that contains only $V_I$ and 
assigns a set of affinitive attributes to each edge that
minimizes the total edge cost. 

\vspace{.5ex}
We will simply denote $\cost(T)$ as
$C(E_T)$ + $I(\overline{V_T})$, the sum of
total edge cost $C(E_T)$ and penalty
$I(\overline{V_T})$.
\eat{
We remark that $C(E_T)$ readily
captures the likelihood the
tree $T$ exists given the
selected affinity attributes, by
setting it as %$\sum_{e\in E_T}C(e, F_T(e))$
$\Pi_{e\in E_T}C(e, F_T(e))$.
Our techniques apply to such models.
}

\begin{example}
Given interested nodes $V_I$ = $\{v_1, v_4, v_5, v_7, v_{11}\}$
in Fig.~\ref{exa-motivate}, where
the interestingness $I(v_7)$ = $0.5$,
and $I(v_{11})$ = $0.3$. The cost
of $T_1$ can be computed as the sum of
the following:
(1) $C((v_1,v_2), \texttt{location})$
refers to the normalized Euclidean distance between
coordinates $v_1.\texttt{location}$ and  $v_2.\texttt{location}$,
similarly for $C((v_i,v_{i+1}), \texttt{location})$ $(i\in[4,6])$);
(2) $C((v_i, v_{i+1}), \texttt{gender})$ = $0$ ($i={2,3}$),
and (3) the cost $I(\overline{V_{T_1}})$
= $I(\{v_{11}\})$ = $0.3$.
The cost of $T_2$ can be computed similarly.
\eat{
\warn{show cases of 1. same topology but different attributes give different
similarity; 2. same attributes but different edges give different meaning.
Give intuition. Properly compare a. community; and b. Steiner tree. link here
in experimental study.}
}
\end{example}


%\stitle{Remarks}. Attributed backbones 
%\reviseS{In the attribute-driven backbone, each edge carries a set of affinitive attributes to suggest the possible "reason" for the connectivity. Whereas, in the traditional community  
%detection problems, either only a set of nodes or a structure with topological constraints is returned, such as $(k,r)$-core in \cite{zhang2017engagement}. The attribute-driven backbone provides the explainability and does not require users to pose explicit topological constraints.}

\vspace{-3ex}
\subsection{Backbone Discovery Problem}
\label{sec-problem}

\vspace{-1ex}
\stitle{Problem statement}. Given a graph $G$
with node set $V$,
node interestingness measure $I$, edge cost
model $C$, and a set of interested nodes $V_I\subseteq V$
with a root node $r$,
the problem of attribute-driven backbone
detection problem (\abd) is to compute a valid attribute-driven backbone $T$ with root $r$
and has a minimum cost \eat{$C(T)$} $\cost(T)$.
More specifically, it solves:

\setlength{\abovedisplayskip}{3pt}
\setlength{\belowdisplayskip}{3pt}

\begin{equation*}
\label{eqn_ABD_primal}
\begin{alignedat}{3}
&\textrm{minimize} \quad && \sum_{e\in E}C(e,F_T(e))x_e+\sum_{U \subseteq V}I(U)z_U\\
&\textrm{subject to} \quad &&\sum_{e \in \delta(S)}x_e + \sum_{U \supseteq S}z_U \geq 1 \quad && \forall S \subseteq V - \{r\},\\
&\quad &&x_e,z_U \geq 0 &&\forall e \in E, U \subseteq V.
\end{alignedat}
\end{equation*}

where the Boolean variable $x_e$ denotes whether edge ${e \in E}$ is in the backbone $T$; $I(U) = \sum_{v \subseteq U}I(v)$, $z_U$ denotes whether a node set ${U \subset V}$ not containing root $r$ is spanned by the backbone $T$, and $\delta(S)$ denotes the edges having exactly one endpoint in $S\subseteq V\setminus\{r\}$.

\begin{theorem}
\label{thm-hardness}
The decision version of \abd problem is
\NP-complete. It is \APX-hard as an optimization
problem.
\end{theorem}

\begin{proofS}
We show the following.
(1) To see that \abd is in \NP, we provide an \NP problem that guesses a
backbone with associated affinitive attributes, and verify its cost
in polynomial time. The hardness can be verified by
a reduction from minimum Steiner tree, a
well-known \NP-hard problem.
The hardness of approximation follows
from an approximation
ratio preserving reduction from
minimum Steiner tree  which is also
\APX-hard~\cite{approx03}.
\end{proofS}

The \APX-hardness of \abd suggests that
no polynomial-time algorithm can approximate
the optimal backbone for
arbitrary approximation ratio.
Despite the hardness, we show that
it is within reach in practice.
We introduce an approximation
algorithm, and a
faster heuristic when $C$ defined as a
probabilistic model, in
Sections~\ref{sec-approx} and
~\ref{sec-em}, respectively.

\stitle{\warn{online backbone discovery problem statement}}







\vspace{-1ex}
%%%%%%%%%%%%%%%%%%% Section 8 %%%%%%%%%%%%%%%%%%%%%%%
\section{Problem Definition}
\label{sec-def}




%%% problem definition

\vspace{-1ex}
%%%%%%%%%%%%%%%%%%% Section 8 %%%%%%%%%%%%%%%%%%%%%%%
\section{Main Part}
\label{sec-main}


%\vspace{-1ex}
%%%%%%%%%%%%%%%%%%% Section 8 %%%%%%%%%%%%%%%%%%%%%%%
\section{Experiment}
\label{sec-expt}



\section {Conclusion}

We have introduced attributed backbones, 
which explicitly incorporate affinitive attributes 
to interpret connectivity in attributed networks. 
We have developed both approximation and 
fast heuristics, to cope with backbone detection 
in networks with small number of attributes, and 
with large attribute size and no explicitly specified 
edge cost model, respectively. 
Our experimental study verified that 
these algorithms can efficiently identify 
interpretable critical structures that 
cannot be identified by conventional 
community detection. 
A future work is to generalize 
our methods to identify 
attribute-driven graphs 
under constraints such as density. \warn{add more about online part.} 



%%
%% The next two lines define the bibliography style to be used, and
%% the bibliography file.
\bibliographystyle{ACM-Reference-Format}
\bibliography{sample-base}

%%
%% If your work has an appendix, this is the place to put it.
\appendix

\section{Research Methods}

\subsection{Part One}

Lorem ipsum dolor sit amet, consectetur adipiscing elit. Morbi
malesuada, quam in pulvinar varius, metus nunc fermentum urna, id
sollicitudin purus odio sit amet enim. Aliquam ullamcorper eu ipsum
vel mollis. Curabitur quis dictum nisl. Phasellus vel semper risus, et
lacinia dolor. Integer ultricies commodo sem nec semper.

\subsection{Part Two}

Etiam commodo feugiat nisl pulvinar pellentesque. Etiam auctor sodales
ligula, non varius nibh pulvinar semper. Suspendisse nec lectus non
ipsum convallis congue hendrerit vitae sapien. Donec at laoreet
eros. Vivamus non purus placerat, scelerisque diam eu, cursus
ante. Etiam aliquam tortor auctor efficitur mattis.

\section{Online Resources}

Nam id fermentum dui. Suspendisse sagittis tortor a nulla mollis, in
pulvinar ex pretium. Sed interdum orci quis metus euismod, et sagittis
enim maximus. Vestibulum gravida massa ut felis suscipit
congue. Quisque mattis elit a risus ultrices commodo venenatis eget
dui. Etiam sagittis eleifend elementum.

Nam interdum magna at lectus dignissim, ac dignissim lorem
rhoncus. Maecenas eu arcu ac neque placerat aliquam. Nunc pulvinar
massa et mattis lacinia.

\end{document}
\endinput
%%
%% End of file `sample-acmsmall.tex'.
